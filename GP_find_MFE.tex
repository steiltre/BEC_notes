%        File: GP_find_MFE.tex
%     Created: Tue Apr 19 08:00 AM 2016 C
% Last Change: Tue Apr 19 08:00 AM 2016 C
%

\documentclass[a4paper]{article}

\title{Gross-Pitaevskii Mean-Field Equation Derivation}
\date{}
\author{}

\usepackage{amsmath}
\usepackage{amsthm}
\usepackage{amssymb}
\usepackage{esint}

\newtheorem{theorem}{Theorem}[section]
\newtheorem{corollary}{Corollary}[section]
\newtheorem{proposition}{Proposition}[section]
\newtheorem{lemma}{Lemma}[section]
\newtheorem*{claim}{Claim}
\newtheorem*{problem}{Problem}
%\newtheorem*{lemma}{Lemma}
\newtheorem{definition}{Definition}[section]

\newcommand{\R}{\mathbb{R}}
\newcommand{\N}{\mathbb{N}}
\newcommand{\C}{\mathbb{C}}
\newcommand{\Z}{\mathbb{Z}}
\newcommand{\supp}[1]{\mathop{\mathrm{supp}}\left(#1\right)}
\newcommand{\lip}[1]{\mathop{\mathrm{Lip}}\left(#1\right)}
\newcommand{\curl}{\mathrm{curl}}
\newcommand{\la}{\left \langle}
\newcommand{\ra}{\right \rangle}
\renewcommand{\vec}[1]{\mathbf{#1}}

\newenvironment{solution}{\emph{Solution.}}

\begin{document}
\maketitle
We will attempt to derive the mean field equations for the Gross-Pitaevskii case using just the general forms for the modulated energy and modulated
stress-energy tensor. This will closely follow Sylvia's work with more detail present.

\section{The PDE}
We will be considering the Gross-Pitaevskii equation
\begin{equation} \label{GP}
  i \partial_t u = \Delta u + \frac{1}{\varepsilon^2} u (1-|u|^2)
\end{equation}

We will let $v$ be the solution to our mean-field equations, which we must still find.

\section{Modulated Energy}
We will define the modulated energy to be of the form
\begin{equation} \label{eqn:mod_energy_form}
  E(u) = \frac{1}{2} \int \chi \left( |\nabla u - i uN_\varepsilon v|^2 + \frac{1}{2\varepsilon^2} (1-|u|^2)^2 + f(|u|^2) g(\psi) \right)
\end{equation}
where $N_\varepsilon$ is the number of vortices, $\chi$ is a smooth cutoff function, $v$ is the mean field equation we wish to find, $\psi$ is some
function of the mean field equation, and $f$ and $g$ are some functions.

We need to take the derivative of $E$ with respect to time. Before we begin with that, we will start by looking at individual pieces of the energy.

\begin{align}
  | \nabla u - i u N_\varepsilon v|^2 &= (\nabla u - i u N_\varepsilon v, \nabla u - i u N_\varepsilon v) \nonumber \\
  &= (\nabla u, \nabla u) - 2N_\varepsilon v ( \nabla u, iu ) + N_\varepsilon^2 |v|^2 (u,u) \nonumber \\
  &= |\nabla u|^2 - 2 N_\varepsilon v j_\varepsilon + N_\varepsilon^2 |v|^2 |u|^2 \nonumber \\
  &= |\nabla u|^2 - 2N_\varepsilon v j_\varepsilon + N_\varepsilon^2 |v|^2 - N_\varepsilon^2 (1 - |u|^2)|v|^2
  \label{eqn:mod_grad}
\end{align}

We now take derivatives of some smaller pieces.

\begin{align}
  \partial_t j_\varepsilon &= \partial_t (\nabla u, iu) \nonumber \\
  &= ( \nabla u_t, iu) + ( \nabla u, i u_t) \nonumber \\
  &= \nabla (u_t, iu) - (u_t, i \nabla u) + (\nabla u, iu_t) \nonumber \\
  &= \nabla (u_t, iu) + (\nabla u, iu_t) + (\nabla u, iu_t) \nonumber \\
  &= \nabla (u_t, iu) + 2 (\nabla u, iu_t) \nonumber \\
  &= \nabla q + V
  \label{eqn:deriv_current}
\end{align}
where we define $q = (u_t, iu)$ and $V=2(\nabla u, iu_t)$ to be the velocity.

\end{document}


