%        File: Thomas-Fermi_profile_facts.tex
%     Created: Wed Sep 14 05:00 PM 2016 C
% Last Change: Wed Sep 14 05:00 PM 2016 C
%

\documentclass[a4paper]{article}

\title{Facts About Thomas-Fermi Profile }
\date{}
\author{Trevor Steil}

\usepackage{amsmath}
\usepackage{amsthm}
\usepackage{amssymb}
\usepackage{esint}

\newtheorem{theorem}{Theorem}[section]
\newtheorem{corollary}{Corollary}[section]
\newtheorem{proposition}{Proposition}[section]
\newtheorem{lemma}{Lemma}[section]
\newtheorem*{claim}{Claim}
\newtheorem*{problem}{Problem}
%\newtheorem*{lemma}{Lemma}
\newtheorem{definition}{Definition}[section]

\newcommand{\R}{\mathbb{R}}
\newcommand{\N}{\mathbb{N}}
\newcommand{\C}{\mathbb{C}}
\newcommand{\Z}{\mathbb{Z}}
\newcommand{\supp}[1]{\mathop{\mathrm{supp}}\left(#1\right)}
\newcommand{\lip}[1]{\mathop{\mathrm{Lip}}\left(#1\right)}
\newcommand{\curl}{\mathrm{curl}}
\newcommand{\la}{\left \langle}
\newcommand{\ra}{\right \rangle}
\renewcommand{\vec}[1]{\mathbf{#1}}

\newenvironment{solution}{\emph{Solution.}}

\begin{document}
\maketitle

\section{Introduction}

The Gross-Pitaevskii equation
\begin{equation}
  i \partial_t u - \Delta u + \frac{1}{\varepsilon^2} ( V(x) + |u|^2 ) u = 0
  \label{eqn:GP}
\end{equation}
with potential $V(x) \to \infty$ as $|x| \to \infty$ has the associated energy

\begin{equation}
  E_{\varepsilon, V} (u) = \int_{\R^2}^{} \frac{| \nabla u|^2}{2} + \frac{1}{\varepsilon^2} \left( V(x) \frac{|u|^2}{2} + \frac{|u|^4}{4} \right)
  \label{eqn:energy_functional}
\end{equation}

Let $M(u) = \|u\|_{L^2}$. For any $m>0$, there exists at least one positive ground state $\eta \equiv \eta_{\varepsilon, m} : \R^2 \to \R^+$
satisfying
\begin{equation}
  E_{\varepsilon, V}(\eta) = \inf \{ E_{\varepsilon, V}(g), g \in H^1( \R^2, \C), M(g) = m \}
  \label{eqn:ground_state_min}
\end{equation}
and also satisfying the Euler-Lagrange equation
\begin{equation}
  -\Delta \eta + \frac{1}{\varepsilon^2} ( V + \eta^2 ) \eta = \frac{1}{\varepsilon^2} \lambda \eta
  \label{eqn:ground_state_Euler_Lagrange}
\end{equation}
for some $\lambda$.

In the limit $\varepsilon \to 0$, we get
\[ \eta^2 \to \rho_{TF} \quad \text{in } L^2(\R^2) .\]
$\rho_{TF}$ is known as the Thomas-Fermi profile and is given by
\[ \rho_{TF}(x) := (\lambda_0 - V)^+(x) \]
where $\lambda_0$ is uniquely determined to satisfy
\[ \int_{\R^2}^{} (\lambda_0 - V(x))^+ dx = m .\]

\section{Some Properties of the Ground State}

\begin{proposition}[6] Proposition 6 says $\eta^2$ is close to $\rho_{TF}$ in $L^2(\R^N)$, and on sets away from where $\rho_{TF} = 0$, they are close in $L^\infty$ (both
differences are bounded by $\varepsilon^{2/3}$). Also, $\| \nabla \eta^2 \|_{L^\infty}$ is bounded away from the set where $\rho_{TF}=0$.
\end{proposition}

\section{Ignat-Millot}
In the Ignat-Millot paper, the ground state satisfies the Euler-Lagrange equation
\[ \begin{cases}
    \varepsilon^2 \Delta \eta_\varepsilon + (a(x) - \eta_\varepsilon^2) \eta_\varepsilon = 0 &\text{ in } \R^2 \\
    \eta_\varepsilon > 0 &\text{ in } \R^2
\end{cases} \]

\end{document}


